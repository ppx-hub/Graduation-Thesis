\chapter{总结讨论}
\section{数据对结果的影响}
\par
在Spiking-SSD中用到了$\mathbf{Data-Base\; Normalization}$的方法,这依赖于训练集中的部分数据,即使采用了0.9~0.99分位点避免一些特殊的离群值,但仍然要保证训练集和测试集服从同一分布,
否则得到的预测结果是不合理的。
\section{影响模型结果的几个因素}
\subsection{时间步长}
\par
在$2.2$节中提到,在仿真时间步长无限长的情况下,脉冲发放率才可以和模拟神经元的值进行近似,因此如果时间步长不足,则转换后模型的预测准确会受很大的影响。
但是如果追求无损的转换需要较高的时间步长,同样会增加了程序运行的时间。
经过实验,在Spiking-SSD和Spiking-Yolo中,时间步长设置为256可以较好地进行预测,这也是在时间和精度的一个折中。
\subsection{原始网络的效果}
\par
转换建立在ANN网络上,因此如果原始网络的训练没有收敛或者是损失没有降到一定的值,会出现ANN网络预测不好的问题,追求ANN近似的SNN自然也不会有好的预测结果,
因此高准确度的SNN网络,如果是通过转换实现的,需要原始的ANN有足够好的性能。
\subsection{转换单元的实现方式}
\par
同样是最大池化操作,现提出的方法有不同的脉冲实现方式。它们在模型上应用的结果也是不同的。
此外,如果不追求纯脉冲神经网络,只模拟部分激活值:另一部分如回归层、预测层仍由ANN实现,
这和纯脉冲操作实现的结果也是不同的。